\documentclass[serif,xcolor=pdftex,dvipsnames,table,hyperref={bookmarks=false,breaklinks}]{beamer}

%%%%%%%%%%%%%%%%
% Change the macros below to configure the title slides
% for your course.
\newcommand{\coursename}{COMPSCI 589}
\newcommand{\instructor}{Benjamin M. Marlin}
\newcommand{\university}{University of Massachusetts Amherst}
\newcommand{\department}{College of Information and Computer Sciences}
%%%%%%%%%%%%%%%%


\newcommand{\settitlecard}[2]{
  \title[\coursename  Lecture #1] 
    {\coursename \\ Lecture #1: #2}
     \author[\instructor]{\instructor}
     \institute[\university]{
     \department\\
     \university
   }
\date{}
}

\newcommand{\maketitlepage}{
  \begin{frame}
  \titlepage
  \center{
    %If you use the slides unmodified, retain the attribution below
    \tiny{Slides by Benjamin M. Marlin (marlin@cs.umass.edu). \\
    \vspace{-1em}Created with support from National Science Foundation Award\# IIS-1350522. 
    %If you modify the slides, please retain the alternate attribution below
    %\tiny{Based on slides by Benjamin M. Marlin (marlin@cs.umass.edu). \\    
    %\vspace{-1em}Created with support from National Science Foundation Award\# IIS-1350522. 
    }                                              
  }  
  \end{frame}
}

\AtBeginSection[]
{
  \begin{frame}<beamer>{Outline}
    \tableofcontents[currentsection,subsectionstyle=hide]
  \end{frame}
}


\newcommand{\cut}[1]{}

\newcommand{\iconbox}[4]{
  \only<#1-#2>{
    \begin{columns}[T]
      \column{0.5in}
           \includegraphics[width=0.5in]{#3}
       \column{3.7in}
            #4
    \end{columns}
    \medskip
    \medskip
    \medskip
  }
}

\mode<presentation>{
  \usepackage{../beamertheme589theme}
  \setbeamercovered{invisible}
}

\mode<handout>{
  \usepackage{../beamertheme589theme}
  \setbeamercovered{transparent}
}


\usepackage[english]{babel}
\usepackage[latin1]{inputenc}
\usepackage{times}
\usepackage[T1]{fontenc}
\usepackage{amsmath}
\usepackage{amssymb}
\usepackage[noend]{algorithmic}
\usepackage{algorithm}
\usepackage{listings}

\renewcommand\mathfamilydefault{\rmdefault}

\newcommand{\setA}{\mathcal{A}}
\newcommand{\setB}{\mathcal{B}}
\newcommand{\setS}{\mathcal{S}}
\newcommand{\setV}{\mathcal{V}}
\DeclareMathOperator*{\union}{\bigcup}
\DeclareMathOperator*{\intersection}{\bigcap}
\DeclareMathOperator*{\Val}{Val}
\newcommand{\mbf}[1]{{\mathbf{#1}}}
\DeclareMathOperator*{\argmax}{arg\,max}
\DeclareMathOperator*{\argmin}{arg\,min}
\DeclareMathOperator*{\sign}{sign}
\newcommand{\deriv}[2]{\frac{\partial{#1}}{\partial{#2}}}


\settitlecard{17}{Mixture Models}

\begin{document}

\maketitlepage

\section{Mixture Models}
\subsection{Foo}

\begin{frame}[t]{Mixture Models}

\begin{itemize}
\item A mixture model is a probabilistic clustering model that is the 
unsupervised analogue of the Bayes Optimal Classifier where the unknown 
assignment of data cases to clusters take the place of the known class labels. 

\pause\item We let $\mbf{x}_i$ be a data case and $z_i\in\{1,...,K\}$ be the 
index of the cluster data case $i$ belongs to. $z_i$ is often called the 
mixture indicator variable or the latent class.

\pause\item Each cluster $k$ specifies it's own distribution over the feature 
vectors $P(\mbf{X}=\mbf{x}|Z=k)$ 

\pause\item We also have a discrete distribution $P(Z=k)=\theta_k$, which 
describes the prior probability that a data case belongs to cluster $k$.
\end{itemize}
\end{frame}

\begin{frame}[t]{Data Distribution}

\begin{itemize}
\item The joint  distribution of the features and the mixture indicator 
variable is:

$$P(\mbf{X}=\mbf{x},Z=k)=\pause P(\mbf{X}=\mbf{x}|Z=k)P(Z=k)$$

\pause\item In clustering, we don't know what the right value of the mixture 
indicator variable is a priori, but we can marginalize it away to obtain a 
probability distribution on the feature vector only:

$$P(\mbf{X}=\mbf{x})=\pause \sum_{k=1}^KP(\mbf{X}=\mbf{x}|Z=k)P(Z=k)$$

\end{itemize}
\end{frame}

\begin{frame}[t]{Mixture Component Distributions}

To define a specific mixture model, we need to define the form of 
$P(\mbf{X}=\mbf{x}|Z=k)$. Some common choices include:

\begin{itemize}
\pause \item Bernoulli: $\displaystyle \prod_{d=1}^D 
\theta_{dk}^{[x_d=1]}(1-\theta_{dk})^{[x_d=0]}$

\pause \item Independent Gaussian: $\displaystyle \prod_{d=1}^D 
\mathcal{N}(x_d;\mu_{dk}, \sigma^2_{dk})$

\pause \item Multivariate Gaussian: $\displaystyle  
\mathcal{N}(\mbf{x};\mu_{k},\Sigma_k)$

\end{itemize}
\end{frame}

\begin{frame}[t]{Learning}

\begin{itemize}
\item Given a data set $\mathcal{D}=\{\mbf{x}_i\}_{i=1:N}$, we can learn the
mixture model parameters by maximizing the log probability of the data give 
the parameters:

$$\mathcal{L} = \sum_{i=1}^N 
\log\left(\sum_{k=1}^KP(\mbf{X}_i=\mbf{x}_i|Z=k)P(Z=k)\right)$$


\pause\item While we can do this directly using gradient-based optimization, 
it's often faster to use a special algorithm called \textit{Expectation 
Maximization}.

\end{itemize}
\end{frame}

\begin{frame}[t]{Expectation Maximization for Gaussian Mixture Models}

\begin{itemize}
\item[E-Step:] In the first step of the algorithm, we compute the 
probability that each data case belongs to each cluster using Bayes rule. These 
probabilities are often called the responsibilities. 

\Large
$$\displaystyle r_{ik} = P(Z_i=k|\mbf{x}_i)=\pause 
\frac{\theta_k \mathcal{N}(\mbf{x};\mu_{k},\Sigma_k)}{\sum_{k'=1}^K\theta_{k'} 
\mathcal{N}(\mbf{x};\mu_{k'},\Sigma_{k'})}$$



\end{itemize}
\end{frame}



\begin{frame}[t]{Expectation Maximization for Gaussian Mixture Models}

\begin{itemize}

\item[M-Step:] In the second step, we update the parameters 
using responsibility weighted averages.

\Large
\pause $$\theta_k = \frac{\sum_{i=1}^N r_{ik}}{N}, \;\;\;\;\;\;
\mu_k = \frac{\sum_{i=1}^N r_{ik}\mbf{x}_i }{\sum_{i=1}^N r_{ik}}$$

\pause $$\Sigma_k = \frac{\sum_{i=1}^N 
r_{ik}(\mbf{x}_i-\mu_k)^T(\mbf{x}_i-\mu_k) }{\sum_{i=1}^N r_{ik}}$$

\end{itemize}
\end{frame}



\begin{frame}[t]{A Special Case}

Suppose we fix $\theta_k=1/K$ and $\Sigma_k = I$. In this case we have:
\pause
$$\mathcal{N}(\mbf{x};\mu_{k},\Sigma_k) = 
\frac{1}{|2\pi I|}\exp(-\frac{1}{2}||\mu_k-\mbf{x}_i||_2^2)$$
and we obtain the following special case of the EM algorithm for multivariate 
Gaussians:

{\Large
\pause
$$\displaystyle r_{ik} = 
\frac{\exp(-\frac{1}{2}||\mu_k-\mbf{x}_i||_2^2)}{\sum_{k'=1}^K\exp(-\frac{1}{2}
||\mu_{k'}-\mbf{x}_i||_2^2)}$$

\pause
$$\mu_k = \frac{\sum_{i=1}^N r_{ik}\mbf{x}_i }{\sum_{i=1}^N r_{ik}}$$
}
\pause
This is often referred to as soft K-means.

\end{frame}

\begin{frame}[t]{Trade-Offs}

\begin{itemize}
\item We can see that the original K-Means algorithm performs hard assignments 
during clustering, and implicitly assumes all clusters will have an equal 
number of points assigned as well as a unit covariance matrix. 

\pause\item EM for Mixtures of Gaussians relaxes all of these assumptions. The 
objective still has multiple local optima, but EM also produces a guaranteed 
non-decreasing sequence of objective function values.

\pause\item EM can also be used with any component densities/distributions to 
customize the model to a given data set.

\pause\item As with K-Means, initialization is important, but the same 
heuristics can be applied. There are similar issues with interpreting output 
and selecting $K$.

\end{itemize}
\end{frame}

\begin{frame}[t]{Choosing K}

\begin{itemize}
\item The Elbow Method: Simple, only requires one fit per value of K. Requires manual assessment of plot. Works for K-Means and Mixture Models.

\pause\item Cross-validation: Requires multiple fits per value of K. Automatic selection of best K. 
Works for GMMs, but often fails for K-Means.
\end{itemize}
\end{frame}


\end{document}
